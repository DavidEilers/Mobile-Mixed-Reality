

\section{Einleitung}
Diese Datei enthlt die Anleitung zur Nutzung der Vorlage fr verschiedene
Typen von Arbeiten. Sie ist vorrangig fr Studenten (und auch wissenschaftliche
Mitarbeiter) gedacht, welche ihre Arbeiten bzw. Publikationen mit \LaTeX{}
erstellen wollen. Dabei wurden auch die Richtlinien des Corporate Design
der Technischen Universitt Chemnitz bercksichtigt, soweit sie sich
ohne grere Probleme in \LaTeX{} realisieren lassen.

Diese Vorlage ist fr folgende Dokumente konzipiert, kann aber bei geringen
Modifikationen auch darber hinaus eingesetzt werden:
\begin{itemize}
\item Hausarbeiten
\item Studienarbeiten
\item Diplomarbeiten
\item Praktikumsberichte
\item Proseminare
\item Oberseminare
\item Hauptseminare
\item sonstige Seminare
\item Belege
\item Studien
\end{itemize}

In den folgenden Kapiteln dieser Anleitung wird ein berblick ber die
Verwendung der Vorlage, Zweck der Dateien und typische Anwendungsflle
gegeben.

\textbf{Hinweis:} Diese Anleitung ist \textbf{keine} Einfhrung
in \LaTeX. Dazu sei auf das Kursangebot des URZ bzw. auf weiterfhrende
Literatur verwiesen. Auch erhebt diese Vorlage \textbf{nicht} den Anspruch,
da jedes damit erstellte Dokument innerhalb der TU-Chemnitz grundstzlich
in Form, Umfang und Aufbau anerkannt wird. Studenten sollten dies
grundstzlich vor der Verwendung anhand der fr sie gltigen Studien-
und Prfungsordnung prfen und darber hinaus mit dem fr sie
zustndigen Professor bzw. Betreuer klren.
\section{Grundkonzept}
Alle auf Basis dieser Vorlage erstellten Dokumente verwenden das sogenannte
{\scshape Koma-Script}-Paket. Dieses Paket wurde entwickelt um \LaTeX{} den
europischen Anforderungen (insbesondere Deutschland) anzupassen.
Diese Anforderungen umfassen u.a.
\begin{itemize}
\item Papierformate
\item verschiedene Sprachen
\item verschiedene Datumsformate
\end{itemize}
Fr weitere Details sei auf die Dokumentation zu {\scshape Koma-Script}
verwiesen.
