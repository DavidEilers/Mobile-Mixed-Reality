In diesem Praktikum war es unsere Aufgabe eine augmented reality App zu entwickeln,
in der zwei Spieler auf zwei Geräten im lokalen Netzwerk gegeneinander
Vier gewinnt spielen können. Dabei sind wir bei der Wahl der AR-Umgebung auf mehrere Sackgassen
gestoßen, haben aber mit AR-Core eine ausreichende Lösung gefunden. Die App wurde hauptsächlich in C++ geschrieben.
Um dies auf Android zu ermöglichen nutzten wir das Android NDK (native development kit). Damit ist es möglich
den für Android üblichen Jave-Code über das Java Native Interface (JNI) mit dem C++ Code kommunizieren
zu lassen. Nicht nur der grafische Teil der App sondern auch die Netzwerkfunktionen wurden mit C++ umgesetzt.
Lediglich die Activities und Menüelemente wurden in Java implementiert.
Beim Testen der App stießen wir auf mehrere Probleme. Zum einen sind Smartphones noch nicht flächendeckend
AR-Core-fähig. Ältere Geräte sind dadurch für Tests ausgeschlossen. Zum anderen ist der Umgang mit dem Android-Emulator
noch schwierig und teilweise garnicht möglich, wenn der PC keine Virtualisierung unterstützt.
Zum Testen der Netzwerkfunktionen kam deshalb ein Python-Skript zum Einsatz.
Die Entwicklung der Netzwerkschnittstelle war insgesamt gesehen aber unkompliziert im Vergleich zum grafischen Teil.
Die App war am Ende des Praktikums Funktionsfähig und kann auf AR-Core-fähigen Smartphones ausgeführt werden.
Für die Zukunft wäre es jedoch Vorteilhafter die Anwendung entweder vollständig in Java oder vollständig in C++
zu entwickeln. Das JNI ist kompliziert und langsam und dadurch hätte man auch auf eine native avtiviy setzen können.
Auf diese Weise wäre der eigentliche Code der App vollstädig in C++ gewesen. Die App ist aber auch
problemlos in Java umsetzbar und im Context von Android ist Java als Programmiersprache die unkompliziertere Wahl.
