\section{NDK}
Für NDK wäre es zu empfehlen, direkt eine Native Activity zu nutzen.
Dabei ist zu bedenken, dass NDK keine UI Elemente hat, somit muss alles selbst
gerendert werden. Es wäre also zu empfehlen, mindestens auf eine UI Library zu setzten
wie beispielsweise Nuklear oder ImGUI oder auf ein komplettes Rendering-Framework,
wie beispielsweise Raylib.

\section{ARCore}
Für ArCore sollte man entweder auf die Java API(oder Unity-/Unreal Plugin) setzen.
Wenn man es in C++ benutzten möchte, wäre zu empfehlen einen Wrapper zu bauen, der die
C-API in eine C++ API wandelt, die ähnlich zu der Java-API ist.

\section{Projektstruktur}
Die Projektstruktur unterscheidet zwischen network und arCore. Generell sollte man
die Klassen ein wenig generischer miteinander Verknüpfen.
Dann könnte man einen GameState haben, der von einer Managementklasse bearbeitet wird
und mit einer Netzwerklasse uptoDate gehalten werden könnte und grafisch
angezeigt werden könnte. Zwar wären weitere Zwischendarstellungen notwendig
(wie beispielsweise für Animationen),
aber man könnte dann den geschriebenen Code wie eine Bibliothek benutzen.
