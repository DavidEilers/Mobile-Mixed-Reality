\section{NDK}
Für NDK wäre es zu empfehlen, direkt eine Native Activity zu nutzen.
Dabei ist zu bedenken, dass NDK keine UI Elemente hat. Somit muss alles selbst
gerendert werden. Es wäre also zu empfehlen, mindestens auf eine UI Library zu setzten
wie beispielsweise Nuklear oder ImGUI oder auf ein komplettes Rendering-Framework,
wie beispielsweise Raylib.


\section{ARCore}
Für ArCore sollte man auf die Java API(oder Unity-/Unreal Plugin) setzen oder,
wenn man es in C++ benutzten möchte, wäre es zu empfehlen, einen Wrapper zu bauen, der die
C-API in eine C++ API wandelt, die ähnlich zu der Java-API ist.


\section{Netzwerk}
Durch die Socket-Struktur, welche in eigentlich allen Programmiersprachen zum Einsatz kommt,
ist die Umsetzung einer Netzwerkschnittstelle im Kern unkompliziert. Die Hauptaufgabe ist es darauf
dann eine oder mehrere abstrakte Schichten aufzubauen, die sich nach den Anforderungen des Projekts
richten. Auf diese Weise gelang es mir die Interaktion mit dem Netzwerk stark zu vereinfachen.


\section{Projektstruktur}
Die Projektstruktur unterscheidet zwischen network und arCore. Generell wäre es empfehlenswert, das Projekt stärker wie eine Bibliothek oder Framework zu strukturieren, damit wäre von `TicTacToe` zu `Four in a row` weniger Aufwand notwendig gewesen.
