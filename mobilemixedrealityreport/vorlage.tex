%======================================================================
%	Vorlage
%======================================================================
%	$Id$
%	Matthias Kupfer
%======================================================================
%	Documentclass
%======================================================================
\documentclass[
%	draft,			% Entwurfsmodus: Bilder als Rahmen,
				% berlngen werden deutlich markiert
%	10pt,			%
	11pt,			% KOMA default
%	12pt,			%
	a4paper,		% DIN A4
	twoside,		% Zweiseitig
	german,			%
%----------------------------------------------------------------------
% Die folgenden Befehle stammen aus dem KOMA Paket
	headsepline,		% Linie unter der Kopfzeile
%	headnosepline,		%
%	foodsepline,		% Linie ber Fussnote
	footnosepline,		%
	automark,		% Kolumnentitel lebendig
%	bigheadings,		% default: berschriften gross setzen
	normalheadings,		% berschriften normal setzen
%	smallheadings,		% berschriften eher klein setzen
%	pointlessnumbers,	% Keinen Punkt hinter die letzte Zahl
				% eines Kapitels (auch bei Anhang)
%	chapterprefix,		% Kapiteluberschriften "Kapitel"
	appendixprefix,		% Anhang
%	openright,		% Kapitel auf der rechten (ungeraden) Seite
				% anfangen lassen
	openany,		%
%	cleardoublestandard,	%
	cleardoubleplain,	%
%	cleardoubleempty,	%
	abstracton,		%
	idxtotoc,		% Index soll im Inhaltsverzeichnis auftauchen
	liststotoc,		%
	bibtotoc,		%
%	parskip, 		% parskip-, parskip*, parskip+
% 	halfparskip, 		% halfparskip-, halfparskip* und halfparskip+
%	DIVclassic,
 	BCOR8mm,		%%?? Bindungskorrektur:
				% BCOR<Breite des Bindungsverlustes>
]{scrreprt}

%======================================================================
%	Bearbeitung sowohl mit LaTeX als auch mit pdfLaTeX ermoeglichen
%======================================================================
%\newif\ifpdf
	%\ifx\pdfoutput\undefined
	%\pdffalse 			% we are not running pdflatex
%\else
	%\pdfoutput=1 			% we are running pdflatex
	%\pdftrue
%\fi


%======================================================================
%	Verwendete Pakete
%======================================================================

%\usepackage{babel}		% Sprachen

%% Latex mit deutschen Umlauten:
%% http://www.cs.albany.edu/~herrmann/latex_umlaute/
\usepackage[utf8]{inputenc}   % Eingabe von ,,, erlaubt

\usepackage[T1]{fontenc}	% EC-Schriften verwenden (vs. DC) da 8-Bit
				% EC-Schriften als T1-kodierten CM-Schriften
				% European/Ext.-Computer-Modern-(EC)-Schriften
				% Umlaute, Anfhrungszeichen ...
				% => Umlauten koennen richtig getrennt werden
				% FAQ 5.3.2

\usepackage{ae,aecompl}		% virtuelle-CM-Fonts
				% da EC nicht als PostScript-(Type-1) verfuegbar
				% => keine echten Umlaute im PDF-Dokumen
				%(Problem bei Suche)
				% By loading the ae package (\usepackage{ae}),
				% you loose some characters as mentioned in
				% README.
				% The package aecompl by Denis Roegel restores
				% these characters which are taken from the ec
				% fonts. If you use pdftex, you will get these
				% characters as bitmaps, but this might be
				% better than not having them at all.

%\usepackage{times, mathptm}	% TimesNewRoman Schrift (Acrobat Reader Fonts),
				% dazu braucht man auch den entsprechenden
				% Zeichensatz fr den Math-Mode
%\usepackage{pslatex}		% ? mathematische Formeln mit Standard
				% Postscript Fonts gesetzt
			% Paket     Roman         Serifenlos  Typewriter
%\usepackage{times}	% -----------------------------------------------
			% times     Times         Helvetica   Courier
			% palatino  Palatino      Helvetica   Courier
			% newcent   NewCenturySch AvantGarde  Courier
			% bookman   Bookman       AvantGarde  Courier
			% Diese Schriften sind die Standard-PostScript-Schriften
			% und in jedem Drucker verfgbar

\usepackage{
%	german,			% Deutsche Trennungen (ALTE Rechtschreibung),
				% Anfhrungsstriche und mehr
	ngerman,		% Deutsche Trennungen (NEUE Rechtschreibung),
				% Anfhrungsstriche und mehr
				%
%	acronym,		% Verwaltung von Abkuerzungen
	bibgerm,		% Deutsche Bibliographie
	calc,			% Erweiterung der arithmetischen Funktionen in
				% LaTeX
				% wird verwendet um Titelseite zu zentrieren
	color,			% im Laufenden Text einfach mit \color{Farbe) zwischen den
				% Farben umschalten, wobei Farbe einfach
				% durch z.B. red, blue, black etc. ersetzt wird
				% \textcolor{farbe){Text)
%	epigraph,		% Zitat am Kapitelanfang
%	fancyhdr,		% Kopf- und Fuzeilen von Dokumenten frei
				% gestalten
	fancybox,		% shadowbox, doublebox, ovalbox, Ovalbox
	fancyvrb,		% verbatim Erweiterung:
	float,			% Positionierung von Gleitobjekten genau an der Stelle, wo man
				% 'figure'- oder 'table'-Umgebung die
				% Positionierung [H] gesetzt werden
%	glosstex,		% Glossar und Abkrzungsverzeichnis
	mdwlist,		% compact list: itemize* ..
	scrdate,		% \todaysname
	scrtime,		% \thistime
	scrpage2,		% Kopf- und Fuzeilen flexibel gestalten
				%
%	moreverb,		% verbatim-hnlich: boxedverbatim, listing
%	verbatim,		% Darstellung von "Text, wie er eingegeben wird"
				%
%	lscape,			% Erstellt eine um 90% gedrehte *neue* Seite
%	textcomp,		% Sonderzeichen
%	booktabs,		% Tabellenlinien
%	longtable,		% Tabellen > 1 Seite
%	supertabular,		% Tabellen > 1 Seite
	tabularx,		% Blocksatzspalten
%	ltxtable,		% tabularx + longtable
%	multicol,		% mehrspaltige Zeilen
%	varioref,		% einheitliche Verweise
%	endnotes,		% Fussnoten -> Endnoten
%	rotating,		% sidewaystable und sidewaysfigure
%	natbib,			% Bibliographie ohne Klammer etc.
%	marvosym,		% Euro etc.
}

%\usepackage{pstricks}
%\usepackage{listings}
%\usepackage{wasysym}

%\usepackage[german, first, bottomafter]{draftcopy}

%\usepackage{setspace}	% Durchschu, Zeilenabstand
%\doublespace		% doppelzeilig oder
%\onehalfspacing	% anderthalbzeilig

% Fr schne Darstellung von Algorithmen
%\usepackage[german, algoruled, algochapter]{algorithm2e}
%\usepackage{algorithmic}
%\usepackage[chapter]{algorithm}
%\floatname{algorithm}{Algorithmus}

%======================================================================
%	Bilder, Links
%======================================================================

\usepackage{graphicx}
%\graphicspath{{../img/}}	% Angabe der Pfade, wo die Grafiken liegen;
				% mehrere Pfade sind mglich

\usepackage{thumbpdf}
\usepackage{hyperref}


%======================================================================
%	user.sty
%======================================================================

\usepackage{user}	% Makros in user.sty
			% \epsinc{bild.eps}{scale=1}{Bildunterschrift}
			% \missing{Beweis fehlt noch}
			% \comment{Ein Kommentar}

%======================================================================
%	Einstellungen
%======================================================================

%\typearea{11}		% Satzspiegel neu konstruieren (KOMA)
			%      10pt 11pt 12pt
			% DIV : 8   10   12

\pagestyle{scrheadings}	% Standart  Kopf- und Fuzeile
\setkomafont{pagehead}{\small\scshape}

% ---------------------------------------------------------------------
%\setcounter{secnumdepth}{2}
%\setcounter{chapter}{-1}
\setcounter{tocdepth}{3}

% ---------------------------------------------------------------------
%\sloppy		% weniger Worttrennungen, grere Wortabstnde
\fussy			% viele Worttrennungen, "schnere" Wortabstnde

% ---------------------------------------------------------------------
%\flushbottom			% Ausrichtung der Seitenenden jeweils auf
				% gleicher Hhe

% ---------------------------------------------------------------------
%\sloppypar		% Das hier relaxt die Einstellungen zum Wortabstand
			% extrem. Damit ragen keine Worte ber den rechten
			% Zeilenabstand hinaus. Dafr mu strker auf
			% Wortabstand geachtet werden, der kann dann ziemlich
			% gro werden. Man erhlt aber keine Meldung mehr
			% ber underfull boxes.

%Hiermit kann man das gleiche mit weniger Holzhammer erreichen:
%\setlength{\tolerance}{2000}           % Strafpunkt fr Zeilenumbruch
%\setlength{\emergencystretch}{3pt}     % Soweit drfen einzelne Worte mehr
					% auseinandergezogen werden
%\setlength{\hfuzz}{1pt}                % Macht den rechten Rand um bis zu 1pt
					% flatterig.

% ---------------------------------------------------------------------
%Vermeiden einzelner Zeilen am Ende einer Seite oder oben auf einer neuen Seite
%\clubpenalty10000
%\widowpenalty10000

% ----------------------------------------------------------------------
% neue Umgebungen fr verwendete Stze und Beispiele
\newtheorem{bsp}{Beispiel}[chapter]
\newtheorem{satz}{Satz}[chapter]

%======================================================================
%	includeonly
%======================================================================

\includeonly{		% Gibt an, welche Dateien der include-Befehl
			% tatschlich einfuegen darf.
  metadaten		% Variablen setzen
  ,titel		% Titelseite, Zusammenfassung und Inhaltsverzeichnis
% Alle per include einzulesenden Dateien mssen hier angegeben sein!
  %,anleitung		% Anleitung zur Nutzung der Vorlage
	,main
}

% Aufteilung der Arbeit in folgende Bestandteile in angeg. Reihenfolge
%
% Titelseite
% Bibliographische Beschreibung (Rckseite der Titelseite)
% (Aufgabenstellung)
% Danksagung
% Abstract
% Inhaltsverzeichnis
% Tabellenverzeichnis
% Abbildungsverzeichnis
% Einleitung
% Inhalt
% Zusammenfassung/Ausblick
% (Verzeichnis verwendeter Terme)
% Glossar (Abkrzungsverzeichnis)
% (Index)
% Literaturverzeichnis
% (Thesen)
% (Selbststndigkeitserklrung)
% Anhnge


%======================================================================
% * *   T E X T    S T A R T S    H E R E   * * * * * * * * * * * * * *
%======================================================================

\begin{document}
%======================================================================
%	Metadaten
%======================================================================
%	$Id$
%	Matthias Kupfer
%======================================================================

\newcommand{\dcsubject}{Praktikumsbericht}
% z.B. (Diplom/Studien/Haus)arbeit, Praktikumsbericht, Studie, Beleg,
% (Pro/Haupt/Ober)seminar, Seminar usw.
\newcommand{\dctitle}{MobileMixedReality Report}
\newcommand{\dcsubtitle}{~} % Untertitel, falls erforderlich

\newcommand{\dcauthorlastname}{David Eilers, Oskar Hippe}
\newcommand{\dcauthorfirstname}{}
\newcommand{\dcauthoremail}{name@subdomain.domain}
\newcommand{\dcdate}{\today}

\newcommand{\dcplace}{Chemnitz} % Ort, kann an der TU meist so bleiben
\newcommand{\dcuni}{Technische Universitt \dcplace}
\newcommand{\dcdepart}{Fakultät Informatik} % Fakulttsangabe
\newcommand{\dcprof}{Professur Graphische Datenverarbeitung und Visualisierung} % Angabe der Professur

\newcommand{\dcpruefer}{Prof. Dr. Guido Brunett}% Prfer der Arbeit
\newcommand{\dcadvisor}{Tom Uhlmann M.Sc.}% Betreuer der Arbeit

\newcommand{\dckeywords}{Liste,von,Stichworten,die,als,Schlagworte,geeignet,
sind}

%%======================================================================
% Einstellungen des Hyperref-Paketes
\hypersetup{%
	pdftitle	= {\dctitle}, %
	pdfsubject	= {\dcsubject, \dcdate}, %
	pdfauthor	= {\dcauthorfirstname~\dcauthorlastname, \dcauthoremail}, %
	pdfkeywords	= {\dckeywords}, %
	pdfcreator	= {pdfTeX with Hyperref and Thumbpdf}, %
	pdfproducer	= {LaTeX, hyperref, thumbpdf}, %
	% weitere PDF-Einstellungen in hyperref.cfg
}
%%======================================================================
 % Variablen, \hypersetup, etc.

%======================================================================
%	Titelseite (Zusammenfassung und Inhaltsverzeichnis)
%======================================================================

%======================================================================
%	Titelseite
%======================================================================
%	$Date:$
%	$Revision:$
%	Matthias Kupfer
%======================================================================

%%======================================================================
%% Schmutztitel
%%======================================================================
%\extratitle{
%	\usekomafont{sectioning}\mdseries
%	\begin{center}
%		\Huge \dcsubject\\[1.5ex]
%		\hrule
%		\vspace*{\fill}
%		\includegraphics{TUC_deutsch_einzeile_CMYK}
%	\end{center}
%}

%%======================================================================
%% Titelkopf
%%======================================================================
\titlehead{
	\vspace*{-1.5cm}
	% Schriftfamilie wie alle berschriften, aber nicht fett
	\usekomafont{disposition}\mdseries
	\begin{center}
		\raisebox{-1ex}{\includegraphics[scale=1.4]{TUC_deutsch_einzeile_CMYK}}\\
		\hrulefill \\[1em]
		{\Large\dcdepart}\\[0.5em]
		\dcprof
	\end{center}
	\vspace*{1.5cm}
}

%%======================================================================
%% Subjekt
%%======================================================================
\subject{\bf\Huge\dcsubject}


%%======================================================================
%% Titel
%%======================================================================
\title{\sf\Large
	\dctitle
	\\
	\dcsubtitle
}

%%======================================================================
%% Autor des Dokumentes
%%======================================================================
\author{\dcauthorfirstname~\dcauthorlastname}

%%======================================================================
%% Ort, Datum
%%======================================================================
\date{\dcplace, den \dcdate
}

%%======================================================================
%% Publishers
%%======================================================================
\publishers{
	{\parbox{\textwidth-8em}{
		\begin{tabbing}
			{\bf Betreuer:}\quad\=\kill
			{\bf Prüfer:}	\>\dcpruefer\\
			{\bf Betreuer:}	\>\dcadvisor
		\end{tabbing}
	}}
}

%%======================================================================
%% bibliografische Angaben
%%======================================================================
\lowertitleback{
\textbf{\dcauthorlastname, \dcauthorfirstname}\\
\dctitle\\
\dcsubject,~\dcdepart\\
\dcuni,~\ifcase\month\or
  Januar\or Februar\or März\or April\or Mai\or Juni\or
    Juli\or August\or September\or Oktober\or November\or Dezember\fi
    ~\number\year
}

%%======================================================================
%% maketitle
%%======================================================================

\maketitle

%%======================================================================
%% Danksagung
%%======================================================================
% \thispagestyle{empty}
% \null\vfil
% \begin{center}
% \usekomafont{disposition}\textbf{Danksagung}
% \vspace{-.5em}\vspace{\parsep}
% %%
% %% Hier steht der Text fr die Danksagung
% %%
% \end{center}
% \par\vfil\null
% \cleardoubleemptypage

%%======================================================================
%%      Kurzfassung / Abstract
%%======================================================================
%%\def\abstractname{Abstract} 	% Wenn der Text "Zusammenfassung" erscheinen
				% soll, dann mu dies auskommentiert werden

%%\begin{abstract}
%%In diesem Praktikum war es unsere Aufgabe eine augmented reality App zu entwickeln,
in der zwei Spieler auf zwei Geräten im lokalen Netzwerk gegeneinander
Vier gewinnt spielen können. Dabei sind wir bei der Wahl der AR-Umgebung auf mehrere Sackgassen
gestoßen, haben aber mit AR-Core eine ausreichende Lösung gefunden. Die App wurde hauptsächlich in C++ geschrieben.
Um dies auf Android zu ermöglichen, nutzten wir das Android NDK (native development kit). Damit ist es möglich
den für Android üblichen Jave-Code über das Java Native Interface (JNI) mit dem C++ Code kommunizieren
zu lassen. Nicht nur der grafische Teil der App sondern auch die Netzwerkfunktionen wurden mit C++ umgesetzt.
Lediglich die Activities und Menüelemente wurden in Java implementiert.
Beim Testen der App stießen wir auf mehrere Probleme. Zum einen sind Smartphones noch nicht flächendeckend
AR-Core-fähig. Ältere Geräte sind dadurch für Tests ausgeschlossen. Zum anderen ist der Umgang mit dem Android-Emulator
noch schwierig und teilweise garnicht möglich, wenn der PC keine Virtualisierung unterstützt.
Zum Testen der Netzwerkfunktionen kam deshalb ein Python-Skript zum Einsatz.
Die Entwicklung der Netzwerkschnittstelle war insgesamt gesehen aber unkompliziert im Vergleich zum grafischen Teil.
Die App war am Ende des Praktikums funktionsfähig und kann auf AR-Core-fähigen Smartphones ausgeführt werden.
Für die Zukunft wäre es jedoch Vorteilhafter die Anwendung entweder vollständig in Java oder vollständig in C++
zu entwickeln. Das JNI ist kompliziert und langsam und dadurch hätte man auch auf eine native avtiviy setzen können.
Auf diese Weise wäre der eigentliche Code der App vollstädig in C++ gewesen. Die App ist aber auch
problemlos in Java umsetzbar und im Context von Android ist Java als Programmiersprache die unkompliziertere Wahl.

%%\end{abstract}

%%======================================================================
%%      Inhaltsverzeichnis
%%======================================================================
%\cleardoubleemptypage
\pagenumbering{roman}
\pdfbookmark{Inhaltsverzeichnis}{Inhaltsverzeichnis}
\tableofcontents

%%======================================================================
%%      Abbildungsverzeichnis
%%======================================================================
%\cleardoublepage
%\markboth{Abbildungsverzeichnis}{Abbildungsverzeichnis}
%\listoffigures

%%======================================================================
%%      Tabellenverzeichnis
%%======================================================================
%\cleardoublepage
%\markboth{Tabellenverzeichnis}{Tabellenverzeichnis}
%\listoftables

%%======================================================================
%%      Algorithmenverzeichnis
%%======================================================================
%\renewcommand{\listalgorithmname}{Algorithmenverzeichnis}
%\cleardoublepage
%\addcontentsline{toc}{chapter}{Algorithmenverzeichnis}
%\listofalgorithms

%%======================================================================
%%      Abkuerzungsverzeichnis
%%======================================================================
%\cleardoublepage
%\addcontentsline{toc}{chapter}{Abkrzungsverzeichnis}
%\markboth{Abk"urzungsverzeichnis}{Abk"urzungsverzeichnis}
%\def\listacronymname{Abk"urzungsverzeichnis}
%\printglosstex(acr)

%%======================================================================
%%      Ende
%%======================================================================
%%\cleardoublepage
\pagenumbering{arabic}
\setcounter{page}{1}
 % Titelseite, Zusammenfassung
		% Inhaltsverzeichnis
		% Abkuerzungsverzeichnis
		% -> danach Seitennummerierung bei 1

%======================================================================
%	Inhalt per include
%======================================================================

%

\section{Einleitung}
Diese Datei enthlt die Anleitung zur Nutzung der Vorlage fr verschiedene
Typen von Arbeiten. Sie ist vorrangig fr Studenten (und auch wissenschaftliche
Mitarbeiter) gedacht, welche ihre Arbeiten bzw. Publikationen mit \LaTeX{}
erstellen wollen. Dabei wurden auch die Richtlinien des Corporate Design
der Technischen Universitt Chemnitz bercksichtigt, soweit sie sich
ohne grere Probleme in \LaTeX{} realisieren lassen.

Diese Vorlage ist fr folgende Dokumente konzipiert, kann aber bei geringen
Modifikationen auch darber hinaus eingesetzt werden:
\begin{itemize}
\item Hausarbeiten
\item Studienarbeiten
\item Diplomarbeiten
\item Praktikumsberichte
\item Proseminare
\item Oberseminare
\item Hauptseminare
\item sonstige Seminare
\item Belege
\item Studien
\end{itemize}

In den folgenden Kapiteln dieser Anleitung wird ein berblick ber die
Verwendung der Vorlage, Zweck der Dateien und typische Anwendungsflle
gegeben.

\textbf{Hinweis:} Diese Anleitung ist \textbf{keine} Einfhrung
in \LaTeX. Dazu sei auf das Kursangebot des URZ bzw. auf weiterfhrende
Literatur verwiesen. Auch erhebt diese Vorlage \textbf{nicht} den Anspruch,
da jedes damit erstellte Dokument innerhalb der TU-Chemnitz grundstzlich
in Form, Umfang und Aufbau anerkannt wird. Studenten sollten dies
grundstzlich vor der Verwendung anhand der fr sie gltigen Studien-
und Prfungsordnung prfen und darber hinaus mit dem fr sie
zustndigen Professor bzw. Betreuer klren.
\section{Grundkonzept}
Alle auf Basis dieser Vorlage erstellten Dokumente verwenden das sogenannte
{\scshape Koma-Script}-Paket. Dieses Paket wurde entwickelt um \LaTeX{} den
europischen Anforderungen (insbesondere Deutschland) anzupassen.
Diese Anforderungen umfassen u.a.
\begin{itemize}
\item Papierformate
\item verschiedene Sprachen
\item verschiedene Datumsformate
\end{itemize}
Fr weitere Details sei auf die Dokumentation zu {\scshape Koma-Script}
verwiesen.
 % Datei, welche die Anleitung zur Nutzung der
		    % Vorlage enthlt
\chapter{Einführung}
\section{Zusammenfassung}
Dies ist der Praktikumsbericht zu unserem Teampraktikum. In diesem haben wir
eine AR App entwickelt, in der man das Spiel ´Vier gewinnt´ spielen kann. Zwei Spieler
können über eine direkte Netzwerkverbindung oder lokal auf dem selben Gerät gegeneinander spielen.
Das Spielfeld wird dabei in augmented reality dargestellt. Es kann also mit Hilfe der Kamera
des Smartphones von allen Seiten betrachtet und damit interagiert werden.

\section{Recherche}
Generell war am Anfang nicht klar, ob wir ein VR oder AR Projekt mache. Deswegen haben wir als erstes nach einem Framework gesucht, welches erlaubt VR oder AR Apps zu bauen.
\\
Wir haben mehrere potentielle Lösungen gefunden:

\begin{itemize}
  \item \verb|C#| Xamarin wurde von unserem Betreuer vorgeschlagen.
  \item WebVR in Webassembly
  \item WebXR in Webassembly
  \item OpenXR in Webassembly
  \item ArCore in Android NDK
\end{itemize}

\subsection{Xamarin}
Wir haben Xamarin getestet und leider hat es unter Linux nicht kompiliert. Auch war das AR
Framework sehr klein und es wurde längere Zeit nicht upgedatet.

\subsection{WebVr}
WebVr hat zwar funktioniert, aber die API gilt als deprecated und es gab nur eine nicht
funktionierendes Beispiel unter Webassembly.

\subsection{WebXR}
WebXR ist der geistige Nachfolger von WebVR und integriert neben VR auch AR support.
Leider ist WebXR sehr neu gewesen und wird bis heute nur in Chromium basierten Browsern
supported. Leider stand in Webassembly lediglich eine WebVR API zu verfügung.

\subsection{OpenXR}
Für OpenXR gab es keinen Webassembly support, generell ist WebXR als Webalternative für OpenXR gedacht.

\subsection{ArCore}
Das Testprojekt in ARCore wurde erfolgreich kompiliert.
Da wir in C++ programmieren sollten, mussten wir auf Android NDK zugreifen.
Unser Betreuer hatte schlechte Erfahrungen mit Android NDK gemacht und uns
dementsprechend davon abgeraten.

\subsection{OpenCV}
Im Verlauf des Projekts entstanden einige Schwierigkeiten und es wurde darüber nachgedacht
unter Android NDK auf OpenCV zurückzugreifen, wir haben uns dabei nur sehr oberflächlich
mit Anchor erkennung mit OpenCV auseinander gesetzt, aber nach grober Recherche, wäre der
Aufwand OpenCV zu benutzten, wahrscheinlich größer gewesen als in ArCore.

\subsection{Entscheidung}
Wir haben uns nach der Recherche für ArCore entschieden, da es aktiv von Google entwickelt
wird, kompiliert werden konnte und auch die Demo recht gut funktioniert. 


\chapter{Entwicklung}
\section{Anmerkung}
Die App wurde mit NDK entwickelt. NDK ist das Native development Kit von Android.
Mit NDK ruft Java C Methoden auf. Auch die API von ARCore ist eine C-API, generell wird in Android NDK aber mit C++ programmiert und die Schnittstelle in
\begin{verbatim}
  extern "C"{

  }
\end{verbatim}
gekapselt. Deswegen wird im folgenden oft über C oder C++ geredet.

\section{ArCore}

\subsection{Entwicklung eines Prototypen}
Am Anfang musste ich die Abhängigkeiten im Buildsystem auflösen, um
ArCore verlinken zu können. Zusätzlich musste ich mich mit dem grundlegenden Lebenszyklus von Apps
auseinander setzten und wegen Android NDK damit beschäftigen, wie man
Parameter von einer Java-Funktion an C übergibt.
Ich habe im nachhinen gemerkt, dass es die Möglichkeit gibt direkt in Android eine
´Native activity´ zu erstellen die nur C++ nutzt, aber alle OpenGL ES Code-samples von Google,
sowie das AR-Core C Beispiel nutzen Java mit eingehängten C++ und auch beim erstellen einer
´Native App´ in Android Studio, wird eine Java Activity mit gelinked C++-Code bereitgestellt.
\par
Daraufhin habe ich eine Dummy-
Implementation von ArCore geschrieben, um ArCore zum laufen zu bringen und besser zu verstehen.
Die Dokumentation von Google zu AR-Core war ein wenig kurz gefasst und ich musste dementsprechend immer wieder auf das Beispiel `\verb|hello_ar_c|`, um die Funktionsweise von ArCore zu verstehen.
Lange Zeit zeigte die Kamera kein Bild an, dabei stellte sich heraus das ArCore ein samplerExternalOES anstatt eines einfachen sampler2D benötigt.
\par
Danach merkte ich, dass der Android Emulator nur bedingt OpenGL ES 3.0 supportet,
deswegen musste ich den Code auf OpenGL ES 2.0 porten, was gerade wegen
Vertex-Array-Objects und eigene Sampler-Objects
umschreiben benötigt hat.
\par
Der Android Emulator hatte einen Bug und somit konnte ArCore App nicht
auf die virtuelle Kamera zugreifen. Das Problem bestand auch bei der Google Beispiel App
``\verb|hello_ar_c|``, aber auch in der Java Version.
\par
Da sich auch heraustellte, dass mein Smartphone ArCore nicht supportet, wurde ich darauf
aufmerksam, dass es eine Liste von supporten Geräten gibt. \url{https://developers.google.com/ar/devices}
\par
Zu dem Zeitpunkt schlug unser Betreuer, Herr Uhlmann, vor auf OpenCV zu wechseln, um dementsprechend dem Problem
zu entgehen. Ich schaute mir OpenCV an, aber da die Entwicklung in OpenCV wohl einen sehr viel
größeren Arbeitsaufwand benötigt hätte, entschied ich mich ein ArCore supportetes Gerät zu kaufen.
\par
Auch unser Betreuer, war dementsprechend entgegen kommend und holte sich ein supportes Tablet.

\subsection{Funktionsweise von ArCore}
ArCore ist eine Bibliothek von Google die unter Android ermöglicht AR Applikationen zu entwickeln.
ArCore analysiert dafür den Kamerastream und kann anhand dessen, die Umgebung erkennen.\\
ArCore bietet dafür folgendes:
\\ \\
arCamera: gibt einem Metainformationen über das Kamerabild,sowei eine Projektionsmatrix und Kameramatrix.
\\ \\
arPlane sind Oberflächen die von ArCore erkannt worden sind.(Wird automatisch generiert)
\\ \\
arPoint sind Punkte reellen 3D Raum die erkannt worden sind.(Wird automatisch generiert)
\\ \\
arAnchor wird vom Entwickler erzeugt, arAnchor erwaret eine Bildposition(zumeist Touchposition), an dieser Bildposition versucht ArCore anhand von eigenen Daten (arPlane,arPoint), den Punkt auf dem geklickt wurde zu erzeugen. Wenn das geklappt hat, wird pro Kameraframe versucht diesen Anchor zu tracken und gibt für diesen Anchor eine Transformationmatrix zurück.


\subsection{Rendering}
\subsubsection{Erste Hilfsklassen}
Für das Rendering habe ich am Anfang ein paar Hilfsklassen gebaut:\\ \\
Shader in Shader.cpp für das lesen von Shadern aus Dateien, compilen und linken.\\ \\
objRenderer in objRenderer.cpp für das Rendern von obj-Files. Später wurde die
Funktionalität in die Mesh Klasse überführt.\\ \\
cameraBackground in cameraBackground.cpp, die das Kamerabild auf einen ScreenQuad sampelt.\\ \\
Aus dem \verb|hello_ar_c| von Google habe ich die objLoader.cpp entnommen, um obj-Dateien
laden zu können.
\subsubsection{Weitere Entwicklung}
Nachdem nun ArCore grundlegend lief habe ich einen Würfel erfolgreich geladen und rendern können.
Nachdem ich dann mit einem Klick auf dem Bildschirm einen
Anchor erzeugen lassen konnte und der Anchor richtig getracked wurde.
Habe ich dann ein simples Szenen System implementiert:
\\ \\
Mesh in Mesh.cpp, dieses lädt eine obj-Datei mithilfe von objLoader.cpp und ermöglicht das rendern dieses Meshes über draw().
\\ \\
Node in Node.cpp enthält
\begin{itemize}
  \item Mesh zum rendern
  \item ModelMatrix in Relation zur Elternnode
  \item Kinder, die auch Nodes sind
\end{itemize}
Scene in Scene.cpp, diese enthält die Root-Node.
Im Konstruktor der Scene werden die Nodes an die Root-Node angehängt oder an andere Nodes die schon angehängt wurden. Beim draw, wird das draw der Root-Node aufgerufen und jede Node ruft widderum
die draw Methode ihrer Childnodes auf und übergibt dabei die eigene Modelmatrix.
\\ \\
Damit wurde das Rendering sehr viel leichter und der Code sehr viel aufgeräumter.

\subsection{Probleme mit ArCore}
Das entwickeln mit ArCore verursachte mehrere Schwierigkeiten, dass größte Problem dabei war
die NDK(C) Version von ArCore. Dabei wurde die objektorientierte API in eine C-API
umgewandelt, was dazu führt, dass wenn in der Javaversion  ein Objekt ein anderes managed,
muss in C immer wieder das Objekt in Aufrufen mitgeschliffen werden.
\\
Hier ein Beispiel, um in der C API, die Modelmatrix von einem getrackten Punkt(anchor) zu bekommen:
\begin{verbatim}
ArPose *pose_;
ArPose_create(arSession, nullptr, &pose_);
ArAnchor_getPose(arSession, anchor, pose_);
ArPose_getMatrix(arSession, pose_, glm::value_ptr(modelMatrix));
ArPose_destroy(pose_);
\end{verbatim}
Wie zu sehen, muss erst eine ArPose erstellt werden, die ArCore erzeugt, dann muss man an diese Pose, die Pose des Anchors binden und kann dann die ModelMatrix bekommen und am Ende muss die Pose wieder gelöscht werden. Dabei muss die arSession immer wieder übergeben werden.
Der gleiche Code in Java:
\begin{verbatim}
modelMatrix = anchor.getPose().toMatrix();
\end{verbatim}

Somit hat es recht lange gedauert, bis ich einen guten Überblick über die C-API
hatte. Am Anfang hatte ich die API direkt in der Nativelib.cpp Datei, die die
Schnittstelle zwischen Java und C++ darstellt und habe diese später in arServer.cpp
migriert, was widerrum die Codekomplexität massiv senkte.

\subsection{Game}
Nachdem ich ArCore und ein gute Szenenabstraktion hatte, habe ich noch einen Hitbox detection geschrieben, diese basiert auf \url{https://www.scratchapixel.com/lessons/3d-basic-rendering/minimal-ray-tracer-rendering-simple-shapes/ray-box-intersection}
die mir Herr Uhlmann gab.\\

Danach haben wir unsere beiden Entwicklungsbranches(Netzwerk, ArCore) gemerged, damit konnte ich
dann auf die GameStates zugreifen und dementsprechend für TicTacToe das Feld rendern.
Dabei sind keine größeren Probleme aufgetreten. \\

Da wir jetzt `Vier Gewinnt` implementieren sollten, habe ich die Scene zu einer vererbaren Klasse refactored (in TicTacToeScene.cpp) und daraufhin Vier gewinnt implementiert(FourInARowScene.cpp). \\

Zu guter Schluss habe ich ein Menü eingebaut, dass am Ende des Spiels angibt, ob man
verloren oder gewonnen hat und ein Button, um das Spiel neuzustarten.
Um den Text anzuzeigen habe ich dafür eine TGA Loader geschrieben und musste
erfahren, dass vom Compiler nicht gewährleistet wird, dass structs tightly packed
sind, weshalb die Headerdaten der TGA nicht richtig ausgelesen wurden. Wofür ich mehrere Stunden debuggen musste. Mit dem fertigen Menü, war dann auch die App fertig.


\section{Netzerk}
\subsection{Grundlegende Netzwerkfunktion}
Als erstes stellte sich die Frage welche Art der Datenübertragung für TicTacToe und Vier gewinnt am sinnvollsten ist.
Ich musste mich hier zwischen UDP und TCP unterscheiden. Beide haben Vor- und Nachteile, die für Echtzeitspiele zu beachten sind:
\par
Heutige Onlinespiele verwenden häufig UDP, da dieses Protokoll verbindungslos ist.
Das bedeutet bei einer Übertragung wird ein Paket ohne Absicherung verschickt. Der Sender kann dabei nicht wissen ob das Paket
beim Empfänger angekommen ist oder, falls mehrere Pakete gesendet wurden, ob diese in der richtigen Reinheinfolge beim Empfänger
eingetoffen sind.
Bei kurzen Statusmeldungen in Spielen, wie Positionsupdates von Spielern, ist UDP passend, da diese Updates sehr schnell wieder
durch neuere Informationen obsolet werden. Der Client sollte sich also nicht damit beschäftigen alte Pakete zu `retten` sondern
möglichst immer bereit für das neuste Paket sein.
\par
Es gibt aber verschiedene Gründe TCP in Spielen zu verwenden.
Gerade wenn es sich bei den zu übertragenden Daten um wichtige Statusmeldungen handelt, die nur einmalig kommuniziert werden,
ist es wichtig, dass diese beim Empfänger ankommen. Wenn beispielsweise ein Spieler in einem Spiel beitritt, wäre es fatal
wenn diese Information bei einem der Mitspieler nicht ankommt. Die Spielumgebung wäre nicht mehr gleich für alle Teilnehmer
und das Spiel unter Umständen garnicht spielbar. TCP eignet sich sehr gut für solche Statusmeldungen, da es erst eine Verbindung aufbaut.
Solange diese Verbindung existiert, ist garantiert, dass die gesendeten Pakete eintreffen. Auch die richtige Reihenfolge ist garantiert.
\par
Wenn man sich Spiele wie TicTacToe und Vier gewinnt anschaut, bemerkt man schnell das diese Spiele keine großen Mengen von Updates
der Spielumgebung benötigen. Es sind nur wenige, dafür aber kritische Meldungen erforderlich um die Spiele für beide Teilnehmer
synchron zu halten.
\par
Benötigt werden beispielsweise für TicTacToe:
\newline
Statusmeldung wo ein Spieler sein X oder O setzt.
\newline
Statusmeldung ob jemand gewonnen hat und wer.
\newline
Statusmeldung über den Abschluss des Spiels wenn das Spielbrett voll ist und es keinen Sieger gibt.
\newline
Statusmeldung für den Neustart des Spiels.
\par
Alle diese Meldungen sind per TCP besser umzusetzen, da sie in jedem Fall beim Empfänger ankommen müssen. Sonst wäre das Spiel
nicht mehr synchron und ggf. in einem undefinierten Zustand.
\subsection{Implementierung der Grundfunktionen}
Da wir uns für die Entwicklung für Android NDK entschieden haben, setzte ich die Implementierung in C++ um.
Dafür war es notwendig zu verstehen wie C++ mit der normalerweise unter Android verwendeten Programmiersprache Java
zusammen arbeitet.
Das Java Native Interface (JNI) wird benutzt um Objecte zwischen Java und C++ hin und her zu `senden`. Diese Übertragungen sind allerdings
zeitaufwendig. Da der visuelle Teil mit ARCore von David auch in C++ implementiert wurde, wollte ich das Netzwerk auch in C++ umsetzen.
\par
Zuerst habe ich Sockets verwendet um Rohdaten empfangen und senden zu können.
In unseren Meetings mit Herrn Uhlmann wurde mir dann vorgeschlagen statt Rohdaten Structs zu verschicken.
Nachdem ich das umgesetzt hatte wurde diese Idee in einem weiteren Meeting erweitert zu Message-Klassen, welche Methoden mitbringen
mit denen die Objekte sich selbst in einen Bytestrom übersetzen und sich aus einem solchen Bytestrom wieder rekonstruieren können.
\subsection{Asnychrones Empfangen}
Als nächstes befasste ich mit dem Asynchronen empfangen der Daten. Da man dauerhaft mit einer Funktion auf eine TCP-Verbindung warten muss
um dann die Nachricht zu verarbeiten, macht es Sinn zumindest das Warten auf neue Verbindungen in einem seperaten Thread zu realisieren.
Dieser Thread ruft dann eine Callbackfunktion auf, die dann mit der Nachricht weiterarbeiten kann.
\subsection{Asynchrones Senden}
Das asynchrone Senden hatte ich mir erst im späteren Verlauf des Projektes vorgenommen. Ich dachte das das Senden nicht sonderlich
zeitaufwendig sein würde und den Spielfluss kaum beeinträchtigen könnte. Damit lag ich falsch. Wenn nämlich ein Fehler beim Senden
auftritt oder das Senden länger braucht als sonst, kann das die ganze App anhalten. Das ist natürlich nicht erstrebenswert.
Also realisierte ich eine Warteschlange in die Nachrichten eingefügt werden. Diese versendet dann der Reihe nach ein anderer Thread.
Bei einer langsamen Verbindung wird dann nicht die gesamte App verlangsamt.
\subsection{Abstaktion}
In Spielen gibt es im Netzwerk oft eine Hierarchie, welche regelt wer der Host eines Spiels ist und wer nur als Client beitritt.
In unserem Projekt haben wir die Begriffe Master und Slave gewählt.
Der Master hält den einzig wahren Spielzustand. Die Slaves übernehmen diesen und senden dem Master ihre Spielzüge wenn
sie an der Reihe sind. Wenn der Master den Spielzug als valide einstuft werden entsprechende Änderung
am Spielfeld vorgenommen und diese allen Slaves mitgeteilt.
So ändert ein Slave also nie sein eigenes Spielfeld. Nur der Master kann Änderungen für sich und alle Slaves durchführen.
Dadurch wird es viel einfacher die Spielfelder zwischen allen Spielern synchron zu halten.
\subsection{Konkrete Umsetzung für Spiele}
Um für David den Netzwerkcode einfach zugreifbar zu machen, habe ich Subklassen von Master und Slave für das jeweilige Spiel abgeleitet.
Diese können dann genutzt werden um Spielzüge zu machen, ohne dass man sich um die Netzwerkschnittstelle kümmern muss.
Master und Slave sind dann nochmal in einem Game-Objekt zusamengefasst damit David den Code für Master und Slave
einheitlich aufrufen kann.
\subsection{Testen der Netzwerkfunktionen}
Ursprünglich wollte ich den Emulator nutzen um die Kommunikation zwischen zwei Instanzen der App zu testen.
Der Emulator hat allerdings eine eigene Firewall, welche für mich nicht so offensichtlich zu konfigurieren war.
Zusätzlich hatte ich das Problem das mein eigentlicher Arbeitscomputer durch unglückliche Umstände keine Virtualisierung unterstützt.
Dadurch musste ich für die Emulation immer einen seperaten PC verwenden. Das war mir auf Dauer dann zu aufwendig und so entschied ich mich
ein Python-Skript zum schreiben. Damit sollte man das TicTacToe-Spiel in der Kommandozeile gegen einen Spieler mit der App spielen können.
Das Skript kann die Rolle das Masters und des Slaves einnehmen. Damit gelang es dann auch die Netzwerkfunktionen zu testen.



\chapter{Resümee}
\section{NDK}
Für NDK wäre es zu empfehlen, direkt eine Native Activity zu nutzen.
Dabei ist zu bedenken, dass NDK keine UI Elemente hat, somit muss alles selbst
gerendert werden. Es wäre also zu empfehlen, mindestens auf eine UI Library zu setzten
wie beispielsweise Nuklear oder ImGUI oder auf ein komplettes Rendering-Framework,
wie beispielsweise Raylib.

\section{ARCore}
Für ArCore sollte man entweder auf die Java API(oder Unity-/Unreal Plugin) setzen.
Wenn man es in C++ benutzten möchte, wäre zu empfehlen einen Wrapper zu bauen, der die
C-API in eine C++ API wandelt, die ähnlich zu der Java-API ist.

\section{Netzwerk}
Durch die Socket-Struktur, welche in eigentlich allen Programmiersprachen zum Einsatz kommt,
ist die Umsetzung einer Netzwerkschnittstelle im Kern unkompliziert. Die Hauptaufgabe ist es darauf
dann eine oder mehrere abstrakte Schichten aufzubauen, die sich nach den Anforderungen des Projekts
richten. Auf diese Weise gelang es mir die Interaktion mit dem Netzwerk stark zu vereinfachen.

\section{Projektstruktur}
Die Projektstruktur unterscheidet zwischen network und arCore. Generell wäre es empfehlenswert, das Projekt stärker wie eine Bibliothek oder Framework zu strukturieren, damit wäre von "TicTacToe" zu "Four in a row" weniger aufwand notwendig gewesen.



%======================================================================
%	Glossar
%======================================================================

%\manualmark
%\addcontentsline{toc}{chapter}{Glossar}
%\markboth{Glossar}{Glossar}
%\def\glossaryname{Glossar}
%\printglosstex(glo)
%\cleardoublepage

%======================================================================
%	Literaturverzeichnis
%======================================================================
\manualmark
\markboth{Literaturverzeichnis}{Literaturverzeichnis}
\bibliographystyle{user}
\bibliography{literatur}
%\cleardoublepage

%======================================================================
%	Thesen
%======================================================================
%\chapter{Thesen}
%\include{thesen}

%======================================================================
%	Selbststndigkeitserklrung
%======================================================================
\chapter*{Selbstständigkeitserklärung}

Hiermit erkläre ich, da ich die vorliegende Arbeit
selbstständig angefertigt, nicht anderweitig zu Prüfungszwecken vorgelegt und
keine anderen als die angegebenen Hilfsmittel verwendet habe. Sämtliche
wissentlich verwendete Textausschnitte, Zitate oder Inhalte anderer Verfasser
wurden ausdrcklich als solche gekennzeichnet.\\[2ex]
\dcplace, den \dcdate\\[6ex]
\flushleft
\newlength\us
\settowidth{\us}{-\dcauthorfirstname~\dcauthorlastname-}
\begin{tabular}{p{\us}}\hline
\centering\footnotesize \dcauthorfirstname~\dcauthorlastname
\end{tabular}

%======================================================================
%	Anhang
%======================================================================


%\part*{Anhang}
%\cleardoublepage
\appendix
%\include{anhanga}

\end{document}
